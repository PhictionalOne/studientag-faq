\begin{Huge}
    Bioinformatik-FAQ
\end{Huge}
\begin{block}{Häufig gestellte Fragen zum Studium}

\question{Lernt man im Studium, wie man programmiert?}{
    Ja, aber auf einer sehr eigenständigen Basis. Man bekommt eine Überblick über die Sprache(n), alles andere was darüber hinaus geht muss man sich selbst aneignen.
}

\question{Welche Programmiersprachen macht man da so?}{
    Ist vom Professor abhängig. In den ersten beiden Semestern meistens entweder Java oder Racket, manchmal auch C++.
}

\question{Muss man programmieren können, um das Studium anzufangen?}{
    Nein. Die Vorlesung beginnt absolut bei 0, um allen den Einstieg zu ermöglichen.
}

\question{Muss man gut in Mathe sein?}{
    Man muss kein Mathe-Genie sein, man sollte Mathe aber nicht hassen. Es ist gerade am Anfang viel Mathe.
}

\question{Muss ich Bio 4-stündig gehabt haben, um Bioinformatik zu studieren?}
{
    Nein, Bioinformatik hat mit der klassischen Schulbiologie absolut nichts zu tun. Es ist allerdings hilfreich, wenn du schon mal das Schema einer Eukaryotenzelle gesehen hast (und weißt, was Eukaryoten sind).
}

\question{Stehe ich als Bioinformatiker viel im Labor?}{
    Jein. Das Studium beinhaltet einige Laborpraktika, aber nicht annähernd so viel wie z.B. bei der Chemie oder Biochemie.
}

\question{Was ist der Unterschied zwischen Bio- und Medizininformatik?}{
    Die Bioinformatik beschäftigt sich grob gesagt mit auotmatisierter Verarbeitung von DNA, Molekülstrukturen etc., Medizininformatik geht mehr in Richtung Patientendaten und medizinische Bildverarbeitung.
}

\question{Wie ist die Frauenquote so?}{
    33\%.
}

\question{Gibt es Praktika?}{
    Im normalen Studienverlauf ist kein berufsorientiertes Praktikum vorgesehen, viele arbeiten aber parallel als Werksstudent oder man macht ein Kurzpraktikum in den Semesterferien.
}

\question{Kann man ein Auslandssemster machen?}{
     Klar, geht immer. Tübingen nimmt am ERASMUS-Programm teil, die Organisation ist aber langwierig und man sollte sich früh (ein Jahr vorher) drum kümmern.
}

\question{Wie ist da so der NC?}{
    Gibt es keinen.
}
\end{block}
