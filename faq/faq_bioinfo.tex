\begin{LARGE}
	Bioinformatik-FAQ
\end{LARGE}
\begin{large}
		\begin{itemize}
		\item \textbf{Lernt man im Studium, wie man programmiert?}
		\begin{itemize}
			\item Ja, aber auf einer sehr eigenständigen Basis. Man bekommt eine Überblick über die Sprache(n), alles andere was darüber hinaus geht muss man sich selbst aneignen.
		\end{itemize}
	\end{itemize}
	
	\begin{itemize}
		\item \textbf{Welche Programmiersprachen macht man da so?}
		\begin{itemize}
			\item Ist vom Professor abhängig. In den ersten beiden Semestern meistens entweder Java oder Racket, manchmal auch C++.
		\end{itemize}
	\end{itemize}
	
	\begin{itemize}
		\item \textbf{Muss man programmieren können, um das Studium anzufangen?}
		\begin{itemize}
			\item Nein. Die Vorlesung beginnt absolut bei 0, um allen den Einstieg zu ermöglichen.
		\end{itemize}
	\end{itemize}
	
	\begin{itemize}
		\item \textbf{Muss man gut in Mathe sein?}
		\begin{itemize}
			\item Man muss kein Mathe-Genie sein, man sollte Mathe aber nicht hassen. Es ist gerade am Anfang viel Mathe.
		\end{itemize}
	\end{itemize}

			\begin{itemize}
				\item \textbf{Muss ich Bio 4-stündig gehabt haben, um Bioinformatik zu studieren?}
				\begin{itemize}
					\item Nein, Bioinformatik hat mit der klassischen Schulbiologie absolut nichts zu tun. Es ist allerdings hilfreich, wenn du schon mal das Schema einer Eukaryotenzelle gesehen hast (und weißt, was Eukaryoten sind).
				\end{itemize}
			\end{itemize}
		
		\begin{itemize}
			\item \textbf{Stehe ich als Bioinformatiker viel im Labor?}
			\begin{itemize}
				\item Jein. Das Studium beinhaltet einige Laborpraktika, aber nicht annähernd so viel wie z.B. bei der Chemie oder Biochemie.
			\end{itemize}
		\end{itemize}
	
	\begin{itemize}
		\item \textbf{Was ist der Unterschied zwischen Bio- und Medizininformatik?}
		\begin{itemize}
			\item Die Bioinformatik beschäftigt sich grob gesagt mit auotmatisierter Verarbeitung von DNA, Molekülstrukturen etc., Medizininformatik geht mehr in Richtung Patientendaten und medizinische Bildverarbeitung.
		\end{itemize}
	\end{itemize}

\begin{itemize}
	\item \textbf{Wie ist die Frauenquote so?}
	\begin{itemize}
		\item 33\%.
	\end{itemize}
\end{itemize}

\begin{itemize}
	\item \textbf{Gibt es Praktika?}
	\begin{itemize}
		\item Im normalen Studienverlauf ist kein berufsorientiertes Praktikum vorgesehen, viele arbeiten aber parallel als Werksstudent oder man macht ein Kurzpraktikum in den Semesterferien.
	\end{itemize}
\end{itemize}

\begin{itemize}
	\item \textbf{Kann man ein Auslandssemster machen?}
	\begin{itemize}
		\item  Klar, geht immer. Tübingen nimmt am ERASMUS-Programm teil, die Organisation ist aber langwierig und man sollte sich früh (ein Jahr vorher) drum kümmern.
	\end{itemize}
\end{itemize}

\begin{itemize}
	\item \textbf{Wie ist da so der NC?}
	\begin{itemize}
		\item Gibt es keinen.
	\end{itemize}
\end{itemize}

	\end{large}