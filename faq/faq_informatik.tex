\begin{Huge}
    Informatik-FAQ
\end{Huge}

\begin{block}{Häufig gestellte Fragen zum Studium}

\question{Lernt man im Studium, wie man programmiert?}{
    Ja, aber auf einer sehr eigenständigen Basis. Man bekommt eine Überblick über die Sprache(n), alles andere was darüber hinaus geht muss man sich selbst aneignen.
}

\question{Welche Programmiersprachen macht man da so?}{
    Ist vom Professor abhängig. In den ersten beiden Semestern meistens entweder Java oder Racket, manchmal auch C++.
}

\question{Muss man programmieren können, um das Studium anzufangen?}{
    Nein. Die Vorlesung beginnt absolut bei 0, um allen den Einstieg zu ermöglichen.
}

\question{Muss man gut in Mathe sein?}{
    Man muss kein Mathe-Genie sein, man sollte Mathe aber nicht hassen. Es ist gerade am Anfang viel Mathe.
}

\question{Brauche ich einen eigenen Laptop?}{
    Ist empfehlenswert. Die Anzahl an Rechnern in den Rechnerräumen ist begrenzt und mit dem eigenen Laptop ist man um einiges flexibler. Tipp: Nicht die Gaming-Maschine, maximal 14 Zoll und lange Akkulaufzeit. Betriebssystem vollkommen egal.
}

\question{Wie ist die Frauenquote so?}{
    17\%. Finden wir auch nicht so wirklich toll.
}

\question{Ich zocke total gerne, hab ich das Zeug, um Informatik zu studieren?}{
    Informatik ungleich Zocken. Du musst analytisches Denken entwickeln, Spaß am Ausprobieren besitzen, willensstark sein und keinen Hass auf Mathe haben (den entwickelt man im Studium dann sowieso).
}

\question{Was arbeitet man danach so?}{
    Alle Bereiche der IT-Branche, z. B. Softwareentwicklung und -Beratung, Hardware-Entwicklung, Automatisierung, Automobilindustrie, Unternehmensberatung, Handel, Banken, Versicherungen...
}

\question{Gibt es Praktika?}{
    Im normalen Studienverlauf ist kein berufsorientiertes Praktikum vorgesehen, viele arbeiten aber parallel als Werksstudent oder man macht ein Kurzpraktikum in den Semesterferien.
}

\question{Kann man ein Auslandssemster machen?}{
    Klar, geht immer. Tübingen nimmt am ERASMUS-Programm teil, die Organisation ist aber langwierig und man sollte sich früh (ein Jahr vorher) drum kümmern.
}

\question{Wie ist da so der NC?}{
    Gibt es keinen.
}
\end{block}
