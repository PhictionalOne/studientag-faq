\begin{Huge}
	Informatik-FAQ
\end{Huge}
\begin{block}{Häufig gestellte Fragen zum Studium}
\begin{large}
\begin{itemize}
		\item \textbf{Lernt man im Studium, wie man programmiert?}
		\begin{itemize}
			\item Ja, aber auf einer sehr eigenständigen Basis. Man bekommt eine Überblick über die Sprache(n), alles andere was darüber hinaus geht muss man sich selbst aneignen.
		\end{itemize}
	
		\item \textbf{Welche Programmiersprachen macht man da so?}
		\begin{itemize}
			\item Ist vom Professor abhängig. In den ersten beiden Semestern meistens entweder Java oder Racket, manchmal auch C++.
		\end{itemize}

		\item \textbf{Muss man programmieren können, um das Studium anzufangen?}
		\begin{itemize}
			\item Nein. Die Vorlesung beginnt absolut bei 0, um allen den Einstieg zu ermöglichen.
		\end{itemize}

	
		\item \textbf{Muss man gut in Mathe sein?}
		\begin{itemize}
			\item Man muss kein Mathe-Genie sein, man sollte Mathe aber nicht hassen. Es ist gerade am Anfang viel Mathe.
		\end{itemize}
	

		\item \textbf{Brauche ich einen eigenen Laptop?}
		\begin{itemize}
			\item Ist empfehlenswert. Die Anzahl an Rechnern in den Rechnerräumen ist begrenzt und mit dem eigenen Laptop ist man um einiges flexibler. Tipp: Nicht die Gaming-Maschine, maximal 14 Zoll und lange Akkulaufzeit. Betriebssystem vollkommen egal.
		\end{itemize}

	

		\item \textbf{Wie ist die Frauenquote so?}
		\begin{itemize}
			\item 17\%. Finden wir auch nicht so wirklich toll.
		\end{itemize}

	

		\item \textbf{Ich zocke total gerne, hab ich das Zeug, um Informatik zu studieren?}
		\begin{itemize}
			\item Informatik ungleich Zocken. Du musst analytisches Denken entwickeln, Spaß am Ausprobieren besitzen, willensstark sein und keinen Hass auf Mathe haben (den entwickelt man im Studium dann sowieso).
		\end{itemize}

	

		\item \textbf{Was arbeitet man danach so?}
		\begin{itemize}
			\item alle Bereiche der IT-Branche, z. B. Softwareentwicklung und -Beratung, Hardware-Entwicklung, Automatisierung, Automobilindustrie, Unternehmensberatung, Handel, Banken, Versicherungen...
		\end{itemize}

	

		\item \textbf{Gibt es Praktika?}
		\begin{itemize}
			\item Im normalen Studienverlauf ist kein berufsorientiertes Praktikum vorgesehen, viele arbeiten aber parallel als Werksstudent oder man macht ein Kurzpraktikum in den Semesterferien.
		\end{itemize}



		\item \textbf{Kann man ein Auslandssemster machen?}
		\begin{itemize}
			\item  Klar, geht immer. Tübingen nimmt am ERASMUS-Programm teil, die Organisation ist aber langwierig und man sollte sich früh (ein Jahr vorher) drum kümmern.
		\end{itemize}



		\item \textbf{Wie ist da so der NC?}
		\begin{itemize}
			\item Gibt es keinen.
		\end{itemize}
\end{itemize}
\end{large}
\end{block}