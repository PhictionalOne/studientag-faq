\begin{Huge}
	Kognitionswissenschaft-FAQ
\end{Huge}
\begin{block}{Häufig gestellte Fragen zum Studium}
\begin{large}
\begin{itemize}
	\item \textbf{Lernt man im Studium, wie man programmiert?}
	\begin{itemize}
		\item Ja, aber auf einer sehr eigenständigen Basis. Man bekommt eine Überblick über die Sprache(n), alles andere was darüber hinaus geht muss man sich selbst aneignen.
	\end{itemize}

	
	\item \textbf{Welche Programmiersprachen macht man da so?}
	\begin{itemize}
		\item Ist vom Professor abhängig. In den ersten beiden Semestern meistens entweder Java oder Racket, manchmal auch C++.
	\end{itemize}



	\item \textbf{Muss man programmieren können, um das Studium anzufangen?}
	\begin{itemize}
		\item Nein. Die Vorlesung beginnt absolut bei 0, um allen den Einstieg zu ermöglichen.
	\end{itemize}



	\item \textbf{Muss man gut in Mathe sein?}
	\begin{itemize}
		\item Man muss kein Mathe-Genie sein, man sollte Mathe aber nicht hassen. Es ist gerade am Anfang viel Mathe.
	\end{itemize}



	\item \textbf{Was mache ich nachher damit?}
	\begin{itemize}
		\item Eine wirklich sehr gute Frage - die nicht so einfach zu beantworten ist. Wir haben die Möglichkeit uns im Master weiter zu spezialisieren, Richtung Informatik, Biologie oder was sonst noch so grob passt. Davon hängt dann auch sehr stark die spätere Berufsaussicht ab. Reine Kognitionswissenschaft findet man vor allem in der Forschung, ein weiteres großes Feld ist Maschinelles Lernen, bzw. Künstliche Intelligenz.
	\end{itemize}



	\item \textbf{Wie ist die Frauenquote so?}
	\begin{itemize}
		\item 55\%.
	\end{itemize}



	\item \textbf{Ich kenne mich mit Informatik gar nicht aus. Ist das schlimm?}
	\begin{itemize}
		\item Nein, ist es nicht. Wenn du einigermaßen logisch denken kannst und dir auch Mathe ganz gut liegt, dann kriegst du das auch hin.
	\end{itemize}



	\item \textbf{Eigentlich will ich ja Psychologie studieren, aber mein NC reicht nicht. Kogni ist doch auch was mit Psychologie, oder?}
	\begin{itemize}
		\item Schon, ja. Allerdings ist der Anteil im Bachelor-Studiengang nicht allzu hoch. Besonders am Anfang sind größere Schwerpunkte Informatik und Mathe, das sollte man auf jeden Fall mit in Betracht ziehen. Im weiteren Verlauf des Studiums kann man sich aber weiter in Richtung Psychologie vertiefen.
	\end{itemize}



	\item \textbf{Kann ich danach auf Psychologie wechseln?}
	\begin{itemize}
		\item Das können wir so allgemein nicht beantworten. Generell hängt das sehr stark von der Uni ab, an der man studieren möchte. Das müsste man dann im Zweifelsfall vorher schon mal anfragen.
	\end{itemize}


	\item \textbf{Gibt es Praktika?}
	\begin{itemize}
		\item Im normalen Studienverlauf ist kein berufsorientiertes Praktikum vorgesehen, viele arbeiten aber parallel als Werksstudent oder man macht ein Kurzpraktikum in den Semesterferien.
	\end{itemize}



	\item \textbf{Kann man ein Auslandssemster machen?}
	\begin{itemize}
		\item  Klar, geht immer. Tübingen nimmt am ERASMUS-Programm teil, die Organisation ist aber langwierig und man sollte sich früh (ein Jahr vorher) drum kümmern.
	\end{itemize}



	\item \textbf{Wie ist da so der NC?}
	\begin{itemize}
		\item 1,8 (WS18/19). Das muss aber in den nächsten Jahren nicht zwingend noch so sein, der Wert ändert sich immer.
	\end{itemize}
\end{itemize}


\end{large}
\end{block}