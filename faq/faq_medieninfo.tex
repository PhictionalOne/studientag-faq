\begin{Huge}
    Medieninformatik-FAQ
\end{Huge}

\begin{block}{Häufig gestellte Fragen zum Studium}

\question{Lernt man im Studium, wie man programmiert?}{
    Ja, aber auf einer sehr eigenständigen Basis. Man bekommt eine Überblick über die Sprache(n), alles andere was darüber hinaus geht muss man sich selbst aneignen.
}

\question{Welche Programmiersprachen macht man da so?}{
    Ist vom Professor abhängig. In den ersten beiden Semestern meistens entweder Java oder Racket, manchmal auch C++.
}

\question{Muss man programmieren können, um das Studium anzufangen?}{
    Nein. Die Vorlesung beginnt absolut bei 0, um allen den Einstieg zu ermöglichen.
}

\question{Muss man gut in Mathe sein?}{
    Man muss kein Mathe-Genie sein, man sollte Mathe aber nicht hassen. Es ist gerade am Anfang viel Mathe.
}

\question{Ich mache voll gerne Design und so, kann ich Medieninformatik studieren?}{
    Medieninformatik ist vor allem eines: Informatik. Gestaltung von Nutzeroberflächen gehört zwar zum Studium dazu, ist aber nur ein Teilgebiet. Der andere Teil ist: viel programmieren, viel Mathe.
}

\question{Ich kann schon Photoshop, bringt mir das bei Medieninformatik was?}{
    Nein.
}

\question{Wie ist die Frauenquote so?}{
    27\%.
}

\question{Lerne ich im Studium, wie man Webseiten baut?}{
    In einer Veranstaltung, ja. Die ersten ca. 3 Semester haben viel mit Web und User-Interfaces zu tun, danach verschiebt sich der Schwerpunkt je nach persönlicher Präferenz.
}

\question{Lerne ich, wie man Computerspiele baut?}{
    Die Möglichkeit besteht, ist aber nicht grundlegender Teil des Studiums und kommt, wenn überhaupt, erst im dritten Studienjahr oder später.
}

\question{Was arbeitet man danach so?}{
    Alle Bereiche der IT-Branche, insbesondere Webentwicklung, Entwicklung von Computerspielen, in der Filmindustrie, Automobilbranche und Medizintechnik.
}

\question{Gibt es Praktika?}{
    Im normalen Studienverlauf ist kein berufsorientiertes Praktikum vorgesehen, viele arbeiten aber parallel als Werksstudent oder man macht ein Kurzpraktikum in den Semesterferien.
}

\question{Kann man ein Auslandssemester machen?}{
    Klar, geht immer. Tübingen nimmt am ERASMUS-Programm teil, die Organisation ist aber langwierig und man sollte sich früh drum kümmern.
}

\question{Wie ist da so der NC?}{
    2,7 (WS18/19). Das muss aber in den nächsten Jahren nicht zwingend noch so sein, der Wert ändert sich immer.
}

\end{block}
