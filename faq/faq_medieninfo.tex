\begin{LARGE}
	Medieninformatik-FAQ
\end{LARGE}
\begin{large}
\begin{itemize}
		\item \textbf{Lernt man im Studium, wie man programmiert?}
		\begin{itemize}
			\item Ja, aber auf einer sehr eigenständigen Basis. Man bekommt eine Überblick über die Sprache(n), alles andere was darüber hinaus geht muss man sich selbst aneignen.
		\end{itemize}
	
		\item \textbf{Welche Programmiersprachen macht man da so?}
		\begin{itemize}
			\item Ist vom Professor abhängig. In den ersten beiden Semestern meistens entweder Java oder Racket, manchmal auch C++.
		\end{itemize}
	
		\item \textbf{Muss man programmieren können, um das Studium anzufangen?}
		\begin{itemize}
			\item Nein. Die Vorlesung beginnt absolut bei 0, um allen den Einstieg zu ermöglichen.
		\end{itemize}
	
		\item \textbf{Muss man gut in Mathe sein?}
		\begin{itemize}
			\item Man muss kein Mathe-Genie sein, man sollte Mathe aber nicht hassen. Es ist gerade am Anfang viel Mathe.
		\end{itemize}

	\item \textbf{Ich mache voll gerne Design und so, kann ich Medieninformatik studieren?}
	\begin{itemize}
		\item Medieninformatik ist vor allem eines: Informatik. Gestaltung von Nutzeroberflächen gehört zwar zum Studium dazu, ist aber nur ein Teilgebiet. Der andere Teil ist: viel programmieren, viel Mathe.
	\end{itemize}

\item \textbf{Ich kann schon Photoshop, bringt mir das bei Medieninformatik was?}
\begin{itemize}
	\item Nein.
\end{itemize}

\item \textbf{Wie ist die Frauenquote so?}
\begin{itemize}
	\item 27\%.
\end{itemize}

\item \textbf{Lerne ich im Studium, wie man Webseiten baut?}
\begin{itemize}
	\item In einer Veranstaltung, ja. Die ersten ca. 3 Semester haben viel mit Web und User-Interfaces zu tun, danach verschiebt sich der Schwerpunkt je nach persönlicher Präferenz.
\end{itemize}

\item \textbf{Lerne ich, wie man Computerspiele baut?}
\begin{itemize}
	\item ie Möglichkeit besteht, ist aber nicht grundlegender Teil des Studiums und kommt, wenn überhaupt, erst im dritten Studienjahr oder später.
\end{itemize}

\item \textbf{Was arbeitet man danach so?}
\begin{itemize}
	\item alle Bereiche der IT-Branche, insbesondere Webentwicklung, Entwicklung von Computerspielen, in der Filmindustrie, Automobilbranche und Medizintechnik.
\end{itemize}

\item \textbf{Gibt es Praktika?}
\begin{itemize}
	\item Im normalen Studienverlauf ist kein berufsorientiertes Praktikum vorgesehen, viele arbeiten aber parallel als Werksstudent oder man macht ein Kurzpraktikum in den Semesterferien.
\end{itemize}

\item \textbf{Kann man ein Auslandssemester machen?}
\begin{itemize}
	\item Klar, geht immer. Tübingen nimmt am ERASMUS-Programm teil, die Organisation ist aber langwierig und man sollte sich früh drum kümmern.
\end{itemize}

\item \textbf{Wie ist da so der NC?}
\begin{itemize}
	\item 2,7 (WS18/19). Das muss aber in den nächsten Jahren nicht zwingend noch so sein, der Wert ändert sich immer.
\end{itemize}

\end{itemize}

\end{large}