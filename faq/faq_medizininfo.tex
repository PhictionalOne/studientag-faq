\begin{Huge}
	Medizininformatik-FAQ
\end{Huge}
\begin{block}{Häufig gestellte Fragen zum Studium}
\begin{large}
\begin{itemize}
	\item \textbf{Lernt man im Studium, wie man programmiert?}
	\begin{itemize}
		\item Ja, aber auf einer sehr eigenständigen Basis. Man bekommt eine Überblick über die Sprache(n), alles andere was darüber hinaus geht muss man sich selbst aneignen.
	\end{itemize}
\end{itemize}
	
\begin{itemize}
	\item \textbf{Welche Programmiersprachen macht man da so?}
	\begin{itemize}
		\item Ist vom Professor abhängig. In den ersten beiden Semestern meistens entweder Java oder Racket, manchmal auch C++.
	\end{itemize}
\end{itemize}

\begin{itemize}
	\item \textbf{Muss man programmieren können, um das Studium anzufangen?}
	\begin{itemize}
		\item Nein. Die Vorlesung beginnt absolut bei 0, um allen den Einstieg zu ermöglichen.
	\end{itemize}
\end{itemize}

\begin{itemize}
	\item \textbf{Muss man gut in Mathe sein?}
	\begin{itemize}
		\item Man muss kein Mathe-Genie sein, man sollte Mathe aber nicht hassen. Es ist gerade am Anfang viel Mathe.
	\end{itemize}
\end{itemize}

\begin{itemize}
	\item \textbf{Ich will eigentlich Medizin studieren, aber mein NC reicht nicht. Medizininformatik ist doch auch was mit Medizin, oder?}
	\begin{itemize}
		\item Nein! Man lernt zwar Grundlagen der Anatomie, Histologie und Pathologie, das ist aber keinesfalls Niveau der Medizin und ihr habt auch keinen Kontakt mit Medizinern. Zweck dieser Vorlesungen ist, am Ende so ungefähr zu verstehen wovon der Mediziner redet.
	\end{itemize}
\end{itemize}

\begin{itemize}
	\item \textbf{Wie ist die Frauenquote so?}
	\begin{itemize}
		\item 60\%.
	\end{itemize}
\end{itemize}

\begin{itemize}
	\item \textbf{Was ist der Unterschied zwischen Bio- und Medizininformatik?}
	\begin{itemize}
		\item Die Bioinformatik beschäftigt sich grob gesagt mit auotmatisierter Verarbeitung von DNA, Molekülstrukturen etc., Medizininformatik geht mehr in Richtung Patientendaten und medizinische Bildverarbeitung.
	\end{itemize}
\end{itemize}

\begin{itemize}
	\item \textbf{Gibt es Praktika?}
	\begin{itemize}
		\item Im normalen Studienverlauf ist kein berufsorientiertes Praktikum vorgesehen, viele arbeiten aber parallel als Werksstudent oder man macht ein Kurzpraktikum in den Semesterferien.
	\end{itemize}
\end{itemize}

\begin{itemize}
	\item \textbf{Kann man ein Auslandssemester machen?}
	\begin{itemize}
		\item Klar, geht immer. Tübingen nimmt am ERASMUS-Programm teil, die Organisation ist aber langwierig und man sollte sich früh drum kümmern.
	\end{itemize}
\end{itemize}

\begin{itemize}
	\item \textbf{Was arbeitet man danach so?}
	\begin{itemize}
		\item alle Bereiche der IT-Branche, insbesondere in den vielfältigen Berufsfeldern der medizinischen Informationsverarbeitung und des Gesundheitswesens.
	\end{itemize}
\end{itemize}

\begin{itemize}
	\item \textbf{Wie ist da so der NC?}
	\begin{itemize}
		\item 2,8 (WS 2018/19). Das muss aber in den nächsten Jahren nicht zwingend noch so sein, der Wert ändert sich hier relativ rasch nach oben.
	\end{itemize}
\end{itemize}

\end{large}
\end{block}