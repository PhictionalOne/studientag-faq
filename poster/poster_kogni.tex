	\begin{LARGE}
			Kognitionswissenschaft
		\end{LARGE}
		\begin{exampleblock}{Was ist der Studiengang?}
			Ein sehr interdisziplinärer Studiengang, der einzelne Aspekte der Informatik, (Neuro-)Biologie, Linguistik, Philosophie und Psychologie miteinander verbindet, bzw. auch einzeln behandelt. Alle Fragen, die dem Denken gewidmet sind, finden hier ihren Platz und werden mithilfe der verschiedenen Sichtweisen der unterschiedlichen Disziplinen versucht zu beantworten. Ein Schwerpunktfach gibt es nicht.
		\end{exampleblock}
	
	\begin{block}{Welcher Teil macht wie viel im Studium aus?}
		\begin{figure}[h!]
			\caption{Verteilung der Themenbereiche über das komplette Studium}
		\end{figure}
	\end{block}
	
	\begin{block}{Was macht man in welchem Semester?}
		\begin{figure}[h!]
			\caption{Vorschlag für den Studienverlauf}
		\end{figure}
		Dieser Verlauf ist allerdings nur ein Vorschlag und kein bindender Studienplan.
	\end{block}
